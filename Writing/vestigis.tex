Moreover, since between 60\% and 80 to 90\% of dementias are created by Alzheimer disease\cite{NoAuthor2017, DSM-V-GOOGLESCHOLAR-CITARMILLOR}] it is then believed that the etiology of a substantial fraction of MCIs will also be produced by Alzheimer's disease \cite{DSM-V-GOOGLESCHOLAR-CITARMILLOR}). Consistently with this thesis, several MCI clinical subtypes\footnote{Mild Cognitive Impairment has never been recognized in the DSM up until now. Its name has changed to mild Neurocognitive Disorder, in case the reader wants to dive further in its nosology} have been proposed in the DSM-V in order to respectively capture different prodromal forms of a variety of dementias\cite{Petersen2004183}.



Over the past four decades multiple neuroimaging techniques have been used in order to study Alzheimer's disease: namely CT, sMRI, fMRI, PET and FD. However, none of them provide exclusive information of the disease when it comes down to obtaining imaging biomarkers. Thus, a combination of them is expected to facilitate diagnosis \cite{Rahim2017, Johnson2012} and the development of effective therapies \cite{Johnson2012}. Currently several researchers have found that [        ]. CITA LA REVISIO DEL BONFILL https://www.ncbi.nlm.nih.gov/pubmed/29164603



According to this systematic review \cite{Arbabshirani2017137} of single-subject prediction studies whose aim was to individually predict mental pathology using neuroimage, Alzheimer's disease and MCI were the brain disorders that have generated more research on individual subject prediction or classification, over schizophrenia, depression, autism or ADHD between the period spanding from 1990 and 2015. 
